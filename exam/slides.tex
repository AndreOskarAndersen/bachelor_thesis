\documentclass{beamer}
\usepackage{graphicx}

\begin{document}

\title{2D Articulated Human Pose Estimation
\newline \small Using Explainable Artificial Intelligence}

\author{André Oskar Andersen
\newline \small \texttt{wpr684}}

\institute{Datalogisk Institut, Københavns Universitet}

\date{2021}

\frame{\titlepage}

\begin{frame}
    \frametitle{Introduktion}
    \begin{minipage}{0.5\textwidth}
        \begin{itemize}
          \item<2-> \textit{Articulated Human Pose Estimation} og \textit{explainable artificial intelligence}
          \item<3-> Anvendelse
          \begin{enumerate}
              \item Motion analysis
              \item Augmented reality
              \item Virtual reality
          \end{enumerate}
          \item <4-> Få kilder udforsker pose estimation algoritmer
        \end{itemize}
      \end{minipage} \hfill
      \begin{minipage}{0.45\textwidth}
        \begin{itemize}
            \item <5-> Hvorfor gør brug af XAI?
            \begin{enumerate}
                \item Forbedrer præstation
                \item Bygger tillid
                \item Vi kan lære af modellen
            \end{enumerate}
            \item<6-> Problem definition
            \begin{enumerate}
                \item Implementer Stacked Hourglass af Newell \textit{et al.}
                \item Udforsk Stacked Hourglass
                \item Modificer Stacked Hourglass vha. viden
            \end{enumerate}
        \end{itemize}
      \end{minipage}
\end{frame}

\begin{frame}
    \frametitle{Data}
    % Hvilken datasæt bruger vi?
    % Beskrivelse af datasæt
    % Preprocessing
    \begin{minipage}{0.5\textwidth}
        \begin{itemize}
            \item<2-> 2017 Microsoft COCO datasæt
            \begin{enumerate}
                \item Passer til problemet
                \item State-of-the-art
            \end{enumerate}
            \item<3-> Beskrivelse
            \begin{itemize}
                \item Træning + validering: $69$.000 billeder
                \item Flere personer i hvert billede
                \item Op til $17$ keypoints per person
            \end{itemize}
        \end{itemize}
    \end{minipage}
    \begin{minipage}{0.45\textwidth}
        \uncover<2->{
          \begin{figure}
            \includegraphics[width = 5cm]{C:/Users/André/OneDrive 2/OneDrive/Skrivebord/bachelor_thesis/thesis/entities/coco_example.PNG}
          \end{figure}
        }
    \end{minipage}
\end{frame}

\begin{frame}
    \frametitle{Preprocessing af data}
    \begin{minipage}{0.5\textwidth}
        \begin{itemize}
            \item<2-> Billederne
            \begin{enumerate}
                \item Centrerer billede omkring hver person
                \item Resizer til $256 \times 256$
                \item Trække gennemsnitlig rgb fra
            \end{enumerate}
            \item<3-> Keypoints
            \begin{enumerate}
                \item Indsætter $1$ i et tomt $64 \times 64$ billede
                \item Gaussfilter
                \item $17$ heatmaps
            \end{enumerate}
        \end{itemize}
    \end{minipage}
    \begin{minipage}{0.45\textwidth}
        \uncover<2->{
          \begin{figure}
            \includegraphics[width = 4cm]{C:/Users/André/OneDrive 2/OneDrive/Skrivebord/bachelor_thesis/thesis/entities/crop_img_2.PNG}
          \end{figure}
        }
    \end{minipage}
\end{frame}

\begin{frame}
    \frametitle{Stacked hourglass}
    % Stacked hourglass
    % Hourglass
    % Residual
    \begin{minipage}{1\textwidth}
        \begin{itemize}
            \item<2-> Stacked hourglass
            \item<3-> Hourglass
            \item<4-> Residual module
            \item<5-> Activation- og lossfunction
        \end{itemize}
    \end{minipage}
    \begin{minipage}{1\textwidth}
        \only<2>{
            \begin{figure}
                \includegraphics[width = 9cm]{C:/Users/André/OneDrive 2/OneDrive/Skrivebord/bachelor_thesis/thesis/entities/SHG.PNG}
            \end{figure}
        }
        \only<3>{
            \begin{figure}
                \includegraphics[width = 7cm]{C:/Users/André/OneDrive 2/OneDrive/Skrivebord/bachelor_thesis/thesis/entities/hourglass.PNG}
            \end{figure}
        }
        \only<4>{
            \begin{figure}
                \includegraphics[width = 6cm]{C:/Users/André/OneDrive 2/OneDrive/Skrivebord/bachelor_thesis/thesis/entities/residual.PNG}
            \end{figure}
        }
    \end{minipage}
\end{frame}

\begin{frame}
    \frametitle{Eksperiment og resultat}
    \begin{minipage}{\textwidth}
        \begin{itemize}
            \item<2-> Kun ét hourglass
            \item<3-> Følger ellers Newell \textit{et al.} og Camilla Olsen
            \item<4-> Overfit
            \item<5-> Epoch 47 - Validation PCK accuracy: 0.433. Test PCK accuracy: 0.441
        \end{itemize}
    \end{minipage}
    \begin{minipage}{\textwidth}
        \only<4>{
          \begin{figure}
            \includegraphics[height = 3.5cm]{C:/Users/André/OneDrive 2/OneDrive/Skrivebord/bachelor_thesis/thesis/entities/results_2.PNG}
          \end{figure}
        }
    \end{minipage}
\end{frame}

\begin{frame}
    \frametitle{Fortolkning af modellen 1 - effekt af skip-connections}
    \begin{itemize}
        \item<2-> Påstand: anvendes til at "redde" information
        \item<3-> SHG med skip-connection vs SHG uden skip-connection
        \item<4-> Resultat
    \end{itemize}
    \begin{minipage}{\textwidth}
        \only<2>{
          \begin{figure}
            \includegraphics[height = 3.5cm]{C:/Users/André/OneDrive 2/OneDrive/Skrivebord/bachelor_thesis/thesis/entities/hourglass.PNG}
          \end{figure}
        }
        \only<4>{
          \begin{figure}
            \includegraphics[height = 6cm]{C:/Users/André/OneDrive 2/OneDrive/Skrivebord/bachelor_thesis/thesis/entities/pred_comparisons_2.PNG}
          \end{figure}
        }
    \end{minipage}
\end{frame}

\begin{frame}
    \frametitle{Fortolkning af modellen 2 - Effekt af principal komponenter}
\end{frame}

\begin{frame}
    \frametitle{Fortolkning af modellen 3}
\end{frame}

\begin{frame}
    \frametitle{Modificering af model}
\end{frame}

\begin{frame}
    \frametitle{Diskussion}
\end{frame}

\begin{frame}
    \frametitle{Konklusion}
\end{frame}

\begin{frame}
    \frametitle{Fejl og rettelser}
\end{frame}

\end{document}