\documentclass[./main.tex]{subfiles}

\begin{document}
\section{Introduction}
% Real-world applications
It is common knowledge, that the real-world use of artificial intelligence and machine learning is growing rapidly. With this grow the need for accurate computer vision models is also growing. One widely used usage of computer vision is \textit{human pose estimation}, where a machine learning model is used for estimating the pose of one, or multiple, humans. These models have many real-world applications, such as motion analysis, augmented reality and virtual reality \cite{survey_2}.
\\
\\
% Motivation for XAI
As the complexity of these models has increased, the models have started to work more and more as a "black box", where it can be difficult to understand how the models work and why they work as they do. This can lead to problems such as distrust, redundancy or difficulty with improving the performance of the models \cite{Selvaraju}.
\\
\\
% Problem statement/thesis goals
The goal of this thesis is thus to develop a model for human pose estimation, as well as interpretating the developed model with respect to getting an understanding of how the different parts of the model works, as well as checking for any redundancy in the model.
\\
\\
% Structure of thesis
In the remainder of this thesis, Section \ref{sec:theory} introduces the basic machine learning thery and Section \ref{sec:algorithms} introduces the most important algortihms throughout the thesis. In Section \ref{sec:dataset} the used dataset and its preprocessing is described. We then describe our development of a model and its respective results in Section \ref{experiement}, which is then explored and interpreted in Section \ref{sec:XAI}. We then discuss our approach and results in Section \ref{sec:discussion}. Lastly, we conclude our results in Section \ref{sec:conclusion}.

\end{document}