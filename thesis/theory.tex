\documentclass[main.tex]{subfiles}
\usepackage[capposition=top]{floatrow}
\usepackage{float}
\begin{document}

\section{Machine Learning Theory}
Throughout this section, the theory of machine learning that will be used in this thesis is described and explained.

\subsection{Motivation}
It can be difficult for humans to recognize certain patterns and trends in data. This becomes more difficult the greater the quantity of the data is, which is becomming more and more common with the rapidly growing topic of \textit{Big Data}. For this reason, computers are often used instead of humans to recognize patterns and trends in the data by analyzing the data, which is what is called \textit{Machine Learning}. In this thesis, we will use machine learning in section \textbf{MANGLER REFERENCE} to develop a model to estimate the 2D pose of a single human in an image. Later, in section \textbf{MANGLER REFERENCE}, we will use machine learning to improve our understanding of the model.

\subsection{Machine Learning Paradigms}
Machine learning usually consists of the three following paradigms
\begin{itemize}
    \item \textit{Supervised learning} where the data consists of features and labels. By analyizing the data the algorithm learns to predict the labels given the features \cite{ESL}.
    \item \textit{Unsupervised learning} where the data only consists of features. The algorithm then learns properties of the data, without any provided labels \cite{ESL}.
    \item \textit{Reinforcement learning} where the algorithm learns to perform the action in a given environment that yields the highest reward \cite{PRML}.
\end{itemize}
In this thesis we will make use of supervised learning when developing our model for pose estimation. Later, unsupervised learning is used when we explore our developed model.

\subsection{Evaluation of Machine Learning Models}
\subsubsection{Splitting the dataset}
\subsubsection{Evaluation Metrics for Supervised Machine Learning}

\subsection{Neural Networks}
\subsection{Convolutional Neural Networks}
\subsection{Stacked Hourglass}
\subsubsection{The Residual Modules}
\subsubsection{The Hourglass}
\subsubsection{The Stacked Hourglass}

\subsection{Principal Components Analysis and K-means Clustering}
\subsubsection{Principal Components Analysis (PCA)}
\subsubsection{K-means Clustering}


\end{document}